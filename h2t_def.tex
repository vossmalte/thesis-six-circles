% % % % % % % % % % % % % % % % % % % % % % % % % % % % % % % % % % % %
% H2T Paper Definitions
%
% Please include this file in all H2T papers and use the definitions
% as a common standard.
%
% If something is missing in this file, please extend it, but keep
% it simple and don't pollute it.
%
% % % % % % % % % % % % % % % % % % % % % % % % % % % % % % % % % % % %

% Packages
\usepackage[per-mode=fraction]{siunitx}
\usepackage{comment}
\usepackage[normalem]{ulem}
\usepackage{xspace}

% Units commonly used in robotics
\DeclareSIUnit\px{px}
\DeclareSIUnit\fps{fps}

% Colors
\definecolor{OliveGreen}{RGB}{0,200,25}
\newcommand{\red}[1]{\textcolor{red}{#1}}
\newcommand{\green}[1]{\textcolor{green}{#1}}
\newcommand{\darkgreen}[1]{\textcolor{OliveGreen}{#1}}
\newcommand{\blue}[1]{\textcolor{blue}{#1}}
\newcommand{\orange}[1]{\textcolor{orange}{#1}}

% Abbreviations
\newcommand{\ie}{i.\,e.\, }
\newcommand{\eg}{e.\,g.\, }
\newcommand{\etal}{et\,al.\ }

% H2T
\newcommand{\hht}{\texorpdfstring{H\textsuperscript{2}T}{H2T}\xspace}
\newcommand{\ackhht}{The authors are with the High Performance Humanoid Technologies Lab, Institute for Anthropomatics and Robotics, Karlsruhe Institute of Technology (KIT), Germany}

% Revision macros
\newcommand{\added}[1]{\darkgreen{#1}}
\newcommand{\replaced}[2]{\red{\ifmmode\text{\sout{\ensuremath{#1}}}\else\sout{#1}\fi}\darkgreen{#2}}
\newcommand{\removed}[1]{\red{\ifmmode\text{\sout{\ensuremath{#1}}}\else\sout{#1}\fi}}
\newcommand{\remark}[1]{\blue{#1}}

% Uncomment the following lines for the final version (without revision marks)
% Attention: Always check the final version!
%            Misplaced revision macros might add unwanted spaces.

\begin{comment}
\renewcommand{\added}[1]{#1}
\renewcommand{\replaced}[2]{#2}
\renewcommand{\removed}[1]{}
\end{comment}

% More complex revision macros (for rebuttal letters, subject to extension)
\newcommand{\addedimage}[1]{\fcolorbox{blue}{blue}{#1}}
\newcommand{\removedfootnote}[1]{\footnote{\removed{#1}}}
\newcommand{\removedsubsection}[1]{\subsection{\texorpdfstring{\removed{#1}}{#1}}}
\newcommand{\replacedsubsection}[2]{\subsection{\texorpdfstring{\removed{#1}\added{#2}}{#2}}}
\newcommand{\addedsection}[1]{\section{\texorpdfstring{\added{#1}}{#1}}}

% Uncomment the following lines for the final version (without revision marks)
% Attention: Always check the final version!
\begin{comment}
\renewcommand{\addedimage}[1]{#1}
\renewcommand{\removedfootnote}[1]{}
\renewcommand{\removedsubsection}[1]{}
\renewcommand{\replacedsubsection}[2]{\subsection{#2}}
\renewcommand{\removedsubsection}[1]{}
\end{comment}

