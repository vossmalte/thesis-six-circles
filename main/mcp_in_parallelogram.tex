\chapter{Parallelogramme}

Die Summe zweier benachbarter Winkel in einem Parallelogramm ist $\pi$,
somit gilt $\alpha_{i+1} = \pi/2 - \alpha_i$.
Durch Phasenverschiebung erhalten wir folgende Identität:
\begin{equation*}
    \tan\alpha_{i+1}  = \tan(\pi - \alpha_i)
    = \frac{\sin(\pi - \alpha_i)}{\cos(\pi - \alpha_i)}
    = \frac{\cos(\alpha_i)}{\sin(\alpha_i)}
    = \cot\alpha_i
\end{equation*}
Daraus folgt
\begin{equation*}
    e_i = \sqrt{\tan\alpha_i \tan \alpha_{i+1}} = 1
\end{equation*}
Die Iterationsvorschrift - als vereinfachte Instanz von \Cref{mcp:iteration-equation} - lautet für Parallelogramme:
\begin{equation*}
    u_{i+1}=|u_i - c_i|
\end{equation*}

Zur Analyse der Periodizität des Money-Coutts-Prozesses verwendet \citet{Troub2000}
eine Funktion $f:\mathbb{R}_0^+ \to \mathbb{R}_0^+$, die ein $u_i$ durch zwei Iterationsschritte transformiert:
\begin{equation*}
    u_{i+2}=||u_i-c_i|-c_{i+1}| =: f(u_i)
\end{equation*}
Dabei ist zu beachten, dass 2-Periodizität von $f$ 4-Periodizität des Money-Coutts-Prozesses bedeutet,
da eine Anwendung von $f$ zwei Kreise in gegenüberliegenden Ecken in Relation setzt.
Die Funktion $f$ ist von den Parametern $c_i$ und $c_{i+1}$ - die mit den Seitenlängen des Parallelogramms korrespondieren und strikt positiv sind - abhängig.

\begin{align}
    f(x) & = \begin{cases}
                 |c_1-x-c_2| \text{ für } c_1 > x \\
                 |x-c_1-c_2| \text{ für } c_1 \leq x
             \end{cases} \\
    % TODO: make these colorful like in the graph
         & =\begin{cases}
                % https://www.overleaf.com/learn/latex/Using_colours_in_LaTeX
                \color{red} // TODO \\
                \\
                \\
                \\
            \end{cases}
    \label{parallelogram:cases}
\end{align}

Ihre zwei Tiefpunkte sind bei $(c_i-c_{i+1}|0)$ und $(c_i+c_{i+1}|0)$;
ein lokales Maximum ist bei $(c_i|c_{i+1})$.
Für $\pm\infty$ strebt die Funktion gegen unendlich.
Der Graph der Funktion ist in \Cref{parallogram:function-graph} zu sehen.

\begin{figure}
    % TODO: make a figure of this graph (in matplotlib?)
    \input{figures/f_in_parallelogram.pgf}
    \caption{Der Graph der Funktion $f$, die zwei Schritte des Money-Coutts-Prozesses in einem Parallelogram beschreibt.
        Die Farben korrespondieren zu denen aus \Cref{parallogram:cases}}
    \label{parallogram:function-graph}
\end{figure}

Das Vorzeichen der Differenz $c_i - c_{i+1}$ bestimmt das periodische Verhalten von $f$.
\begin{itemize}
    \item Ist $c_i=c_{i+1}$, so ist $f(x)=x$ für $x\in[0,c_i]$.
    \item Für $c_i>c_{i+1}$ ist $f(f(x))=x$ für $x\in[0,c_i-c_{i+1}]$.
    \item Für $c_i<c_{i+1}$ ist $f(f(x))=x$ für $x\in[c_i,c_i+c_{i+1}]$.
\end{itemize}

Für $x=u_i$, das nicht in einem solchen Intervall liegt, gilt,
dass nach endlich vielen Anwendungen von $f$ auf $x$ das Resultat in einem solchen Intervall liegt.
% TODO: begründe, warum in ein solches Intervall abgebildet wird.
Somit gibt es eine endlich lange Präperiode.

\paragraph{Anmerkungen}
Der Fall $c_i=c_{i+1}$ entspricht einer Raute.
Analog zu \Cref{regular-polygon:chapter} ist der Iterationsschritt (\ref{mcp:iteration-equation}) selbstinvers für \emph{kleine} $u_i$.
% TODO: bessere Formulierung für Vertauschung
Die anderen beiden Fälle sind austauschbar, denn führt man einen Iterationsschritt aus und verwendet von da an $f$,
so sind $c_i$ und $c_{i+1}$ vertauscht.
