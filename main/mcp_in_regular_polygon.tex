\chapter{Regelmäßige Polygone}

Wendet man den Money-Coutts-Prozess auf ein regelmäßiges Polygon $P$ mit $n$ Ecken an, so wird aufgrund der Spiegelsymmetrie von $P$ schnell klar,
dass der Prozess periodisch ist.
Betrachtet man einen Kreis des Prozesses, so sind sein Vorgänger und Nachfolger kongruent.
Daraus folgt, dass der Prozess $n$-periodisch für $n$ gerade respektive $2n$ periodisch ist für $n$ ungerade \citep{Taba2000}.

Wir untersuchen dies algebraisch als Vorbereitung auf nachfolgende Kapitel.
Es gilt: $\alpha_i = \frac{(n-2) \pi}{2n} = \alpha$ für alle $i$, da jeder Innenwinkel eines regelmäßigen Polygons gleich ist.
Ebenso ist $e_i=\tan \alpha$ für alle $i$ gleich.
Ohne Beschränkung der Allgemeinheit können wir annehmen, dass $c_i=1$ ist,
da wir das Polygon so skalieren können, dass alle Seiten die Länge $1$ haben.
Damit können wir \Cref{mcp:iteration-equation} vereinfachen:

\begin{equation}
    u_{i+1}=||
\end{equation}